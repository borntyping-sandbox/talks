\documentclass{beamer}
\usepackage{minted}

\usetheme{Singapore}
\usecolortheme{beaver}

\title{Writing simple tools: Spotter}
\subtitle{\url{http://github.com/borntyping/spotter}}
\author{Sam Clements}

\hypersetup{
    pdftitle={Writing simple tools: Spotter},
    pdfauthor={Sam Clements},
    pdfstartview={Fit},
}

\begin{document}
\begin{frame}
    \titlepage
\end{frame}

\begin{frame}{About Spotter}
    \begin{itemize}
        \item Spotter is a command line tool
        \item It takes a set of filename patterns paired with commands
        \item It waits for those files to change and then runs the command
    \end{itemize}
\end{frame}

\begin{frame}[fragile]{Existing tools}{\texttt{inotify}}
\begin{minted}[frame=single]{bash}
while inotifywait -e close_write <path>; do
    <command>
done
\end{minted}
\begin{minted}[frame=single]{bash}
python -m pyinotify <path> --command <command>
\end{minted}
\end{frame}

\begin{frame}[fragile]{Existing tools}{\url{https://github.com/mynyml/watchr}}
\texttt{watchr} uses a ruby file containing filename patterns and functions:
\begin{minted}[frame=single]{bash}
watchr watch.rb
\end{minted}
\begin{minted}[frame=single]{ruby}
watch(<path>) {system(<command>)}

watch("test/(.*)\.rb") do |md|
    system("ruby test/#{md[1]}.rb")
end
\end{minted}
\end{frame}
    
\begin{frame}{Why make Spotter?}
    All of the existing tools are either short bash one liners or use a full programming language. \vspace{1em}

    With Spotter:
    \begin{itemize}
        \item It's very easy to understand and to read
        \item It doesn't require the use of a programming language
        \item The patterns and commands can be kept in the same place as a project, without being a full script
    \end{itemize}
\end{frame}

\begin{frame}[fragile]{Spotter files}
Watch a file for changes
\begin{minted}[frame=single]{text}
watch: *.txt -> echo "File changed"
\end{minted}
\pause
\vspace{1em}
Use the filename in the command
\begin{minted}[frame=single]{text}
watch: *.txt -> cat "{filename}"
\end{minted}
\end{frame}

\begin{frame}[fragile]{Spotter files}
Running commands at the start and finish
\begin{minted}[frame=single]{text}
start: echo "Started watching files"

watch: *.txt -> echo "File changed"

stop: echo "Finished watching files"
\end{minted}
\end{frame}

\begin{frame}[fragile]{Spotter files}
Defining constants
\begin{minted}[frame=single]{text}
define: <name> -> <value>
\end{minted}
\pause
\vspace{1em}
Run a command when spotter starts and when a file changes
\begin{minted}[frame=single]{text}
define: build -> python build.py

start: {build}

watch: *.html -> {build} --filename "{filename}"
\end{minted}
\end{frame}

\begin{frame}[fragile]{Using spotter to compile latex}{\texttt{spotter .spotter-pdflatex}}
\begin{minted}[frame=single]{text}
# Watch the LaTeX document
watch: *.tex -> pdflatex -draftmode "{filename}"
watch: *.tex -> pdflatex "{filename}"
watch: *.tex -> texcount -brief "{filename}"

# Clean unneeded files on exit
stop: rm -f *.aux
stop: rm -f *.log
stop: rm -f *.out
stop: rm -f *.toc
\end{minted}
\end{frame}

\begin{frame}
    \frametitle{Using spotter to compile latex}
    \framesubtitle{Useful for non-dedicated-\LaTeX{} editors}
    \includegraphics[width=\textwidth]{2013-spotter-and-latex.png}
\end{frame}

\begin{frame}[fragile]{Using spotter to build a static website}
\begin{minted}[frame=single]{text}
# Watch the site for changes
define: build -> ./vdg.py --build-path ../build

# Build the entire site at start
start: {build}

# Rebuild the whole site when templates changes
watch: templates/* -> {build}

# Build individual files when they change
watch: * -> {build} {filename}
\end{minted}
\end{frame}

\begin{frame}
    \titlepage
\end{frame}
\end{document}
